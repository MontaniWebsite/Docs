\documentclass{article}

\usepackage{graphicx} 
\usepackage{hyperref}                         % Pacchetti
\usepackage[italian]{babel}
\graphicspath{ {./images/} }
\usepackage{fancyhdr}
\usepackage{imakeidx}


\makeindex[columns=1, title=Tavola dei contenuti, intoc]

\begin{document}
	
	
	\begin{titlepage}
		\begin{center}
			\huge\textbf{Montani WebSite}\\
			\Large\textbf{5 INB}\\
			\Large \textbf{Specifica dei Requisiti}\\
			\vspace{4cm}
			\large Project Manager: \textbf{Boussoufa Yacine}\\
			\large Data: \textbf{07/02/2022}\\
			\large Versione:\textbf{ 0.1}\\
			\large ID SRS: \textbf{Allegato A}\\
			\large ID Progetto: \textbf{Montani}\\
			
		\end{center}
	\end{titlepage}
	
	\clearpage
	
	\begin{tabular}{ |p{1cm}|p{4cm}|p{3cm}|p{2cm}|  }
		\hline
		\multicolumn{4}{|c|}{Cronologia delle revisioni} \\
		\hline
		ID& Cambiamenti &Data di creazione&Autore\\
		\hline
		001   & Creazione    &25/01/2022&   Camilletti Samuele\\
\hline
	\end{tabular}
	
	\clearpage
	
	\tableofcontents
	\printindex	
	
   

    %_______________________________________________________________________________________________
    %ANALISI PROBLEMA	
	\section{\textbf{Introduzione al documento}}
	\flushleft
	\normalsize
	Il presente documento ha lo scopo di presentare una visione globale della Specifica dei Requisiti del progetto Montani WebSite. La struttura del documento è quella suggerita dallo standard ANSI/IEEE  830 noto come SRS (Software Requirements Specifications).
	\normalsize
	 \subsection{\textbf{Obiettivo}} 
	\flushleft
	\normalsize
	Lo scopo del presente documento è di rappresentare, nel modo più preciso, completo, coerente, non  ambiguo e comprensibile, i requisiti relativi alla ridefinizione delle modalità di fruizione dei servizi scolastici mediante sito web dell'Istituto Tecnico Tecnologico ITT Montani di Fermo.\\ 
	Per specifica dei requisiti si intende l’elicitazione di tutti i requisiti utente e di sistema del sistema informativo senza specificare le metodologie applicate per la risoluzione di essi.
	
	\subsection{\textbf{Campo di applicazione}}
	L'obiettivo del presente progetto è di realizzare un sito web per l'Istituto Tecnico Tecnologico Montani di Fermo. La scuola dispone già di un sito web, il quale non rispetta i requisiti grafici e funzionali definiti dal modello standard di siti web scolastici realizzato dal Team per la Trasformazione Digitale su richiesta del Ministero dell'Istruzione. Il progetto si basa sulla metodologia, gli strumenti e il design system di Designers Italia e a sua volta, contribuisce ad alimentare il design system della Pubblica Amministrazione mettendo a disposizione di tutte le amministrazioni componenti e pattern elaborati. Per ulteriori informazioni consultare \url{https://docs.italia.it/italia/designers-italia/design-scuole-docs/it/master/progetto-siti-web-delle-scuole.html}. Partendo dallo studio della realtà pre-esistente è stato necessario ridefinire le modalità di navigazione e di interfacciamento con l'utente nel sito web. Un'ulteriore campo di applicazione del sito web è l'accessibilità, ovvero la capacità del sistema informativo di erogare informazioni rispettando le linee guida sull’accessibilità degli strumenti informatici secondo quanto descritto nell’articolo 11 della legge n. 4/2004.
	
	
	\subsection{\textbf{Definizioni, acronimi e abbreviazioni}}
	\begin{tabular}{ |p{3cm}|p{8cm}|  }
	\hline
	\textbf{Termine}& \textbf{Definizione}\\
	\hline
	Studente   & Studente che frequenta l'Istituto  \\
	\hline
Genitori & Genitori degli Studenti \\
\hline
Personale tecnico amministrativo & Presidi, Amministratori, Figure Amministrative \\
	\hline
	Insegnanti   &  Professionista nel campo dell'istruzione, operante nell'Istituto.  \\
	\hline
	Visitatore  & Soggetto non correlato all'Istituto in alcun modo se non per attività di orientamento. \\	
	\hline
\end{tabular}
	
	\subsection{\textbf{Sinonimi}}
\begin{tabular}{ |p{3cm}|p{8cm}|  }
	\hline
	\textbf{Termine}& \textbf{Sinonimi}\\
	\hline
	Studente   & Alunno, Allievo  \\
	\hline
	Genitore   &  Parenti, Affidatari \\
	\hline
	Insegnanti  & Docenti\\	
	\hline
	Visitatore   & Studente non iscritto.  \\
	\hline
\end{tabular}
	
	\subsection{\textbf{Struttura del documento e riferimenti}}
	Nella sezione 2 di questo documento verranno illustrate le interfacce di sistema e gli utenti coinvolti.\\
	Nella sezione 3 di questo documento verranno illustrate le specifiche funzionali e non funzionali del sistema.\\
	Questo documento si riferisce al project charter con ID Progetto Montani.\\
	Per la programmazione delle attività di lavoro fare riferimento ad Allegato B e per l'analisi dei rischi all'allegato C.\\
	
	\vspace{1cm}
	
	\section{\textbf{Descrizione generale}}
     Questa sezione descrive le funzionalità generale del prodotto, le interfacce e gli utenti del sistema.

	\subsection{\textbf{Inquadramento}}
	Questo progetto mira a ridefinire il modello di fruizione dei servizi scolastici dell'Istituto Montani di Fermo attraverso il suo sito web.
	
	Il sito web è accessibile da qualsiasi device connesso ad internet, attraverso l'utilizzo di un browser web e digitando l'indirizzo istitutomontani.edu.it.\\
	Al caricamento della pagina verrà mostrata l'homepage, dalla quale si potrà scegliere il facilmente il servizio che l'utente sta cercando. 
	
	\subsubsection{\textbf{Interfaccia utente}}
	L’interfaccia sistema/utente è stata realizzata attraverso un sito web il quale utilizza il CSM Wordpress. WordPress è una piattaforma software di content management system (CMS) che, operando lato server in un database, consente la creazione di un sito Internet formato da contenuti testuali o multimediali, gestibili ed aggiornabili in maniera dinamica; facendo uso di codice HTML CSS e JavaScript. Wordpress mette a disposizione anche plugin e temi per la personalizzazione del sito web. 
	
	\subsubsection{\textbf{Interfaccia software}}
	Come spiegato nella sezione 2.1.1 il sito web è stato realizzato attraverso il software di gestione di contenuti (CMS) Wordpress. Wordpress necessita di un database per la memorizzazione di tutte le informazioni relative al sito e un Web Server Apache.  
	
	\subsubsection{\textbf{Interfaccia hardware}}
	Per la mantenimento del web server e del database viene utilizzata un'unica macchina attiva 24/7 della scuola ITIS Montani. Il dominio (istitutomontani.edu.it) segue le normative relative agli indirizzi istituzionali.
	
	\subsubsection{\textbf{Interfaccia di comunicazione}}
	Non definita.	

	\subsection{\textbf{Definizione degli utenti}}
\begin{tabular}{ |p{3cm}|p{8cm}|  }
	\hline
	\textbf{Tipo}& \textbf{Descrizione}\\
	\hline
	Studente   & Gli studenti possono visionare le informazioni relative ad orari. \\
	\hline
	Genitore   & I genitori possono visionare le informazioni relative ad orari.
	 \\
	\hline
	Amministratori  & Gli amministratori hanno completo accesso alle funzionalità del CSM.\\	
	\hline
	Docente & I docenti hanno accesso al lato gestionale del CSM e wordpress e si occupano di inserire e modifica progettualità. Non possono modificare notizie.
\end{tabular}

	\subsection{\textbf{Requisiti da analizzare in futuro}}
Non definiti.	
\clearpage
	\Large \section{\textbf{Specifiche funzionali e non funzionali del sistema}} 
	\normalsize
	\flushleft
	\subsection{\textbf{Requisiti funzionali}}
	\begin{itemize}
		\item Il sito web deve essere disponibile a tutti i dispositivi dotati di connessione internet e deve prevedere una home di accesso dalle quale venga effettuata una panoramica dei servizi.
		\item Il sito web deve presentare un rifacimento sotto l'aspetto grafico al fin di modernizzare l'interfaccia attuale.
		\item Il sito web deve prevedere funzionalità aggiuntive come l'inserimento delle progettualità o l'aggiornamento degli orari.
	\end{itemize}
	
	\subsubsection{\textbf{Caso d'uso: Connessione alla home page del sito}}
\begin{tabular}{ |p{3cm}|p{8cm}|  }
	\hline
	\multicolumn{2}{|c|}{\textbf{RF - 001}} \\
	\hline
	Specifica& Connessione alla home page del sito\\
	\hline
	Attori& Utente/Operatore/Amministratore\\
	\hline
	Pre-Condizioni& L'attore deve predisporre di un dispositivo connesso alla rete internet e di un software browser\\
	\hline
	Scenario principale& \begin{enumerate}
		\item L'attore apre il broswer
		\item L'attore inserisce nella barra di ricerca l'indirizzo "infomontani.ns0.it:8289"
		\item La pagina home viene caricata con successo
			\end{enumerate}\\
		\hline
	Post-Condizioni& L'attore può visitare la home page del sito ISS Hosting\\
\hline
\end{tabular}
\clearpage
	\subsubsection{\textbf{Caso d'uso: Connessione alle sotto pagine del sito}}
\begin{tabular}{ |p{3cm}|p{8cm}|  }
	\hline
	\multicolumn{2}{|c|}{\textbf{RF - 002}} \\
	\hline
	Specifica& Connessione alle pagine secondario\\
	\hline
	Attori& Utente/Operatore/Amministratore\\
	\hline
	Pre-Condizioni& L'attore deve essere connesso alla home o al sito.\\
	\hline
	Scenario principale& \begin{enumerate}
		\item L'attore apre il menù a tendina posto in alto a sinistra
		\item L'attore sceglie la sotto pagina da visitare
		\item La pagina home viene caricata con successo
	\end{enumerate}\\
	\hline
	Post-Condizioni& L'attore può scegliere di quale servizio usufruire.\\
	\hline
\end{tabular}
	\subsubsection{\textbf{Caso d'uso: Inserimento di nuovi articoli o post}}
\begin{tabular}{ |p{3cm}|p{8cm}|  }
	\hline
	\multicolumn{2}{|c|}{\textbf{RF - 003}} \\
	\hline
	Specifica& Inserimento di nuovi articoli o post\\
	\hline
	Attori& Operatore/Amministratore\\
	\hline
	Pre-Condizioni& L'attore deve aver effettuato il login in wordpress.\\
	\hline
	Scenario principale& \begin{enumerate}
		\item L'attore sceglie "New" dalla barra in alto presente in ogni pagina del sito
		\item L'attore sceglie la tipologia di post da effettuare ed inserisce il testo/media dall'apposito editor.
		\item L'attore clicca il tasto "Publish" a sinistra nell'editor.
	\end{enumerate}\\
	\hline
	Post-Condizioni& Il post/articolo viene pubblicato.\\
	\hline
\end{tabular}
	\subsubsection{\textbf{Caso d'uso: Acquisto di un servizio}}
\begin{tabular}{ |p{3cm}|p{8cm}|  }
	\hline
	\multicolumn{2}{|c|}{\textbf{RF - 004}} \\
	\hline
	Specifica& Acquisto di un servizio\\
	\hline
	Attori& Utente\\
	\hline
	Pre-Condizioni& L'attore deve possedere un account.\\
	\hline
	Scenario principale& \begin{enumerate}
		\item L'attore sceglie una pagina tra Host dedicati, VPS o Game Host.
		\item L'attore sceglie il pacchetto che più desidera tra quelli elencati forniti di tutte le caratteristiche necessarie.
		\item L'attore clicca su BUY IT NOW.
	\end{enumerate}\\
	\hline
	Post-Condizioni& L'utente riceve un'email con le istruzioni per il pagamento e le credenziali di accesso al servizio.\\
	\hline
\end{tabular}
	\normalsize
\flushleft
\vspace{4mm} 
\subsection{\textbf{Requisiti non funzionali}}
In questa sezione si sintetizzano tutte le caratteristiche non funzionali emerse dalla fase di  analisi dei vincoli di diversa natura raccolti durante le interviste e lo studio dell'ambito d'applicazione. Queste caratteristiche definiscono tutte le funzionalità “esterne” al sistema:
\vspace{4mm} \vspace{4mm} 
\begin{tabular}{ |p{3cm}|p{8cm}|  }
	\hline
	\multicolumn{2}{|c|}{\textbf{RNF - 001}} \\
	\hline
	Specifica&Portabilità\\
	\hline
	Metrica & Dispositivi\\
	\hline
	Descrizione&Il sito web deve essere accessibile e visualizzabile da ogni dispositivo con connessione internet ed un browser web. Il sito deve essere portabile anche su altri Web Server esportando il database MySQL.\\
	\hline
\end{tabular}\\
\vspace{4mm} 
\begin{tabular}{ |p{3cm}|p{8cm}|  }
	\hline
	\multicolumn{2}{|c|}{\textbf{RNF - 002}} \\
	\hline
	Specifica&Usabilità \\
	\hline
	Metrica & Tempo di lettura articoli\\
	\hline
	Descrizione&Il sito web deve essere comprensibile e facilmente leggibile. Il tempo medio di lettura di un articolo o post deve essere inferiore ai 2 minuti.\\
	\hline
\end{tabular}\\
\clearpage
\begin{tabular}{ |p{3cm}|p{8cm}|  }
	\hline
	\multicolumn{2}{|c|}{\textbf{RNF - 003}} \\
	\hline
	Specifica &Implementazione\\
	\hline
	Metrica &  -\\
	\hline
	Descrizione&Il sito web deve essere implementato attraverso il CSM Wordpress.\\
	\hline
\end{tabular}\\
\vspace{4mm} 
\begin{tabular}{ |p{3cm}|p{8cm}|  }
	\hline
	\multicolumn{2}{|c|}{\textbf{RNF - 004}} \\
	\hline
	Specifica&Legislativo\\
	\hline
	Metrica &  -\\
	\hline
	Descrizione&Il sito web deve rispettare l'INFORMATIVA sulla PRIVACY GDPR(Regolamento 2016/679) relativo al trattamento dei dati sensibili online.\\
	\hline
\end{tabular}\\
\vspace{4mm} 
\begin{tabular}{ |p{3cm}|p{8cm}|  }
	\hline
	\multicolumn{2}{|c|}{\textbf{RNF - 005}} \\
	\hline
	Specifica&Etico \\
	\hline
	Metrica &  -\\
\hline
	Descrizione&Il sito web deve rappresentare attraverso articoli dedicati e link di approfondimento la posizione presa dall'azienda ISS Hosting in merito al tema climatico. L'azienda utilizza fonti di energia rinnovabile per il sostentamento delle proprie infrastrutture e utilizza rifiuti RAEE per la costruzione di sistemi server.\\
	\hline
\end{tabular}\\
\section{\textbf{Ulteriori specifiche}}
Ulteriori specifiche, se necessarie, verranno aggiunte nelle prossime versioni del seguente documento.
\end{document}
	
    