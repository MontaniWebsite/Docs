\documentclass{article}

\usepackage{graphicx}                       % Pacchetti
\usepackage[italian]{babel}
\graphicspath{ {./images/} }
\usepackage{imakeidx}


\makeindex[columns=1, title=Tavola dei contenuti, intoc]

\begin{document}
	
	
	\begin{titlepage}
		\begin{center}
			\huge\textbf{Montani Website}\\
			\Large\textbf{5 INB}\\
			\Large \textbf{Project Charter}\\
			\vspace{4cm}
			\large Project Manager: \textbf{Boussoufa Yacine}\\
			\large Data: \textbf{07/02/2022}\\
			\large Versione:\textbf{0.1}\\
		\end{center}
	\end{titlepage}
	
	\clearpage
	
	\begin{tabular}{ |p{1cm}|p{4cm}|p{3cm}|p{2cm}|  }
		\hline
		\multicolumn{4}{|c|}{Cronologia delle revisioni} \\
		\hline
		ID& Cambiamenti &Data di creazione&Autore\\
		\hline
		0   & Creazione    &07/02/2022&   Capretti Mattia\\
		\hline
	\end{tabular}
	
	\clearpage
	
	\tableofcontents
	
	
	
	\printindex
	
	%Stampa del titolo, autore e data
	
	
	\textbf{{\fontsize{5mm}{10mm}\selectfont \section{Background} }} 
	\begin{flushleft}
		L'Istituto Tecnico Tecnologico Statale “G. e M. MONTANI” di Fermo ha richiesto al richiesto di effettuare una modernizzazione grafica e funzionale del sito web della scuola, il quale non rispetta i requisiti grafici e funzionali definiti dal modello standard di siti web scolastici realizzato dal Team per la Trasformazione Digitale su richiesta del Ministero dell’Istruzione.
		
	\end{flushleft}
	\vspace{3mm}
	\textbf{{\fontsize{5mm}{10mm}\selectfont \section{Obiettivo del progetto} }} 
	\begin{flushleft}
		L'obiettivo del presente progetto è di realizzare un sito web per l'Istituto Tecnico Tecnologico Montani di Fermo, al fine di risolvere i problemi di accessibilità, riscontrati dalle famiglie degli studenti e dai professori dell'istituto dovuti all'interfaccia grafica poco intuitiva e non più in regola con le direttive nazionali.   
		Il restyling deve prevedere:
		\begin{itemize}
			\item \textbf{Home Page}: miglioramento della home page intento a rendere più intuitivo l' approccio al sito. 
			\item \textbf{pagine degli indirizzi}: miglioramento delle pagine contenenti le informazioni riguardanti gli indirizzi.
			\item \textbf{pagine non funzionanti}: risoluzione dei problemi legati ai link non funzionanti o che indirizzano a pagine non esistenti.
			\item \textbf{privacy}: Consentire solo agli utenti loggati di visualizzare informazioni legate alle ore dei professori.
			
		\end{itemize}
		\vspace{2mm}
		
		\subsection{Progettazione preliminare}
		Il sito Web verrà realizzato attraverso l'utilizzo del CSM Wordpress. Precedentemente alla realizzazione di questo progetto l'Istituto Tecnico Tecnologico statale “G. e M. MONTANI” è già in possesso un sito web di conseguenza questo non è un progetto realizzato ex-novo.
		Per la specifica dei requisiti e relativi casi d'uso consultare \textbf{allegato "A"}.\\ 
		\vspace{2mm} 
	\end{flushleft}
	\vspace{1cm}
	\textbf{{\fontsize{5mm}{10mm}\selectfont \section{Ambito del progetto (SCOPE)} }} 
	\begin{flushleft}
		Definizione dei deliverable:
		\begin{itemize}
			\item Project management: Analisi del sito web in relazioni ai requisiti utente e di sistema.
			\item Documentazione: redazione docunento SRS e project charter
			\item Realizzazione: prototipo grafico del sito.
			\item Test: confronto con il cliente.
		\end{itemize}
		
		Definizione delle milestone:
		\begin{itemize}
			\item Accordo con il cliente per l'interfaccia grafica.
			\item realizzazione interfaccia grafica.
			\item Caricamento dei materiali relativo alle pagine.
			\item Consegna della versione finale del sito.
		\end{itemize}
		\subsection{In ambito}
		Nel progetto è incluso:\\
	    \begin{itemize}
			\item \textbf{Home Page}: miglioramento della home page intento a rendere più intuitivo l' approccio al sito. 
			\item \textbf{pagine degli indirizzi}: miglioramento delle pagine contenenti le informazioni riguardanti gli indirizzi.
			\item \textbf{pagine non funzionanti}: risoluzione dei problemi legati ai link non funzionanti o che indirizzano a pagine non esistenti.
			\item \textbf{privacy}: Consentire solo agli utenti loggati di visualizzare informazioni legate alle ore dei professori.
			\item \textbf{backhand}: .
		\end{itemize}
		\subsection{Fuori ambito}
		Nel progetto non è incluso:
		\begin{itemize}
			\item Implementazione delle procedure di pagamento.
			\item Sistema di registrazione ed accesso per gli utenti.
		\end{itemize}
		\vspace{2mm}
		\subsection{Programmazione della tempistica preliminare}
		\begin{tabular}{ |p{6cm}|p{4cm}|  }
			\hline
			Descrizione del prodotto da consegnare &Data obiettivo\\
			\hline
			redazione documentazione project charter &09/02/2022\\
			\hline
			Realizzazione prototipo interfaccia grafica &09/02/2022\\
			\hline
			confronto con il committente &09/02/2022\\
			\hline
			Debug e consegna &X/X/2022\\
			\hline
		\end{tabular}
		\vspace{3mm}
		\textbf{{\fontsize{5mm}{10mm}\selectfont \section{Criteri di successo} }} 
		Il sito web soddisfare i criteri stabiliti nell'allegato relativo alla specifica dei requisiti:
		\begin{itemize}
			\item Funzionali: esegue coerentemente i requisiti funzionali del progetto.
			\item Non funzionali: rispetta i requisiti non funzionali del progetto.
			\item Esterni: è conforme con le norme sulla privacy attualmente in vigore.
			\item Organizzativi: è conforme alle caratteristiche organizzative dell'area commerciale e marketing dell'azienda.
		\end{itemize}
		\vspace{3mm}
		\textbf{{\fontsize{5mm}{10mm}\selectfont \section{Assunzioni e vincoli iniziali} }} 
		\begin{tabular}{ |p{1cm}|p{2cm}|p{6cm}|  }
			\hline
			ID&Tipo &Descrizione\\
			\hline
			0&Temporale&Scadenza 28/02/2022\\
			\hline
			1&Economico&inesistenza di budget\\
			\hline
			2&Ambientale&Ambiente di sviluppo - CSM Wordpress-joomla\\
			\hline
			3&Legislativo&Politiche relative all'uso delle informazioni degli utenti\\
			\hline
		\end{tabular}
		\clearpage
		\textbf{{\fontsize{5mm}{10mm}\selectfont \section{Team e comunicazione} }}
		\subsection{Definizione degli stakeholders}
		\begin{itemize}
			\item Promotore: Istituto Tecnico Tecnologico Statale “G. e M. MONTANI” di Fermo
			\item Cliente:Istituto Tecnico Tecnologico Statale “G. e M. MONTANI” di Fermo
			\item Beneficiario: Istituto Tecnico Tecnologico Statale “G. e M. MONTANI” di Fermo
			\item Team di progetto:
			\begin{itemize}
				\item Project Manager: Boussoufa Yacine
				\item Analyst: Capretti Mattia
				\item Programmer: Enrique Nucci
				\item Debugger: Camilletti Samuele
			\end{itemize}
			\item Soggetti esterni all'organizzazione legati alle risorse: daniele trasatti
			\item Soggetti condizionanti: famiglie degli studenti, studenti, professori dell'istituto
		\end{itemize}
		\subsection{Parti interessate, Ruoli e responsabilità}
		\begin{tabular}{ |p{4cm}|p{4cm}|p{4cm}|  }
			\hline
			Parte interessata&Ruolo &Responsabilità\\
			\hline
			Boussoufa Yacinea&Project Manager&Alta\\
			\hline
			Capretti Mattia&Analyst&Alta\\
			\hline
			Enrique Nucci&Programmer&Alta\\
			\hline
			Camilletti Samuele&Debugger&Alta\\
			\hline
			Istituto Tecnico Tecnologico Statale “G. e M. MONTANI” di Fermo&Committente&Media\\
        	\hline
        	Studente   & Utente del sito & Bassa. \\
        	\hline
        	Genitore   & Utente del sito & Bassa
        	 \\
        	\hline
        	Amministratori  & Amministratore del sito & Bassa.\\	
        	\hline
        	Docente & Utente del sito & Bassa.\\
        	\hline
		\end{tabular}
		\vspace{3mm}
		\subsection{Piano di comunicazione iniziale}
		\begin{tabular}{ |p{3cm}|p{3.5cm}|p{2cm}|p{1.5cm}|p{2cm}|  }
			\hline
			Comunicazione&Descrizione &Frequenza&Formato&Destinatari\\
			\hline
			Aggiornamento stato con il cliente&Confronto dei progressi con i requisiti utente&Ogni 2 giorni&Smart-working&Daniele Trasatti\\
			\hline
		\end{tabular}
		
		\textbf{{\fontsize{5mm}{10mm}\selectfont \section{Analisi dei rischi} }} 
		Per l'analisi e valutazione dei rischi consultare \textbf{allegato "B"}.
		\begin{tabular}{ |p{1cm}|p{2cm}|p{6cm}|p{2cm}|  }
			\hline
			ID&Tipo &Descrizione&Livello\\
			\hline
			0&temporale&tempo necessario per la realizzazione dell'interfaccia grafica insufficente&Basso\\	
			\hline
			1&Requisiti&Mancata comprensione della richiesta iniziale&Basso\\	
			\hline
		\end{tabular}
		
		\vspace{12mm}
		\textbf{{\fontsize{5mm}{10mm}\selectfont \section{Autorità del progetto} }} 
		Sponsor: istituto tecnico tecnologico statale “G. e M. MONTANI” di fermo\\
		Project Manager: Boussoufa Yacine\\
		Committente: istituto tecnico tecnologico statale “G. e M. MONTANI” di fermo\\
		\vspace{2mm}
		\subsection{Allegati}
		\begin{itemize}
			\item Allegato A: SRS.
			\item Allegato B: Analisi e valutazione dei rischi.
			\item Allegato C: WBS.
		\end{itemize}
		\subsection{Firme}
		Project Manager: Boussoufa Yacine\\
		Committente: istituto tecnico tecnologico statale “G. e M. MONTANI” di fermo
		
	\end{flushleft}
	
	
	%_______________________________________________________________________________
	%fine del documento
\end{document}

