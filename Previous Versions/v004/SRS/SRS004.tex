\documentclass{article}

\usepackage{graphicx} 
\usepackage{hyperref}                         % Pacchetti
\usepackage[italian]{babel}
\graphicspath{ {./images/} }
\usepackage{fancyhdr}
\usepackage{imakeidx}


\makeindex[columns=1, title=Tavola dei contenuti, intoc]

\begin{document}
	
	
	\begin{titlepage}
		\begin{center}
			\huge\textbf{Montani WebSite}\\
			\Large\textbf{5 INB}\\
			\Large \textbf{Specifica dei Requisiti}\\
			\vspace{4cm}
			\large Project Manager: \textbf{Boussoufa Yacine}\\
			\large Data: \textbf{11/02/2022}\\
			\large Versione: \textbf{0.4}\\
			\large ID SRS: \textbf{Allegato A}\\
			\large ID Progetto: \textbf{Montani}\\
			
		\end{center}
	\end{titlepage}
	
	\clearpage
	
	\begin{tabular}{ |p{1cm}|p{4cm}|p{3cm}|p{2cm}|  }
		\hline
		\multicolumn{4}{|c|}{Cronologia delle revisioni} \\
		\hline
		ID& Cambiamenti &Data di creazione&Autore\\
		\hline
		001   & Creazione    &07/02/2022&   Camilletti Samuele\\
		\hline
		002   & Aggiunta Requisiti Funzionali classificati per utente    &08/02/2022&   Camilletti Samuele\\
		\hline
		003   & Modifica dei requisiti dopo revisione con il cliente    &11/02/2022&   Camilletti Samuele\\
		\hline
		004   & Aggiunta dei requisiti dopo revisione con il cliente    &15/02/2022&   Camilletti Samuele\\
\hline
		005   & Aggiunta dei requisiti dopo revisione con il cliente    &16/02/2022&   Camilletti Samuele\\
\hline
	\end{tabular}
	
	\clearpage
	
	\tableofcontents
	\printindex	
	
   

    %_______________________________________________________________________________________________
    %ANALISI PROBLEMA	
	\section{\textbf{Introduzione al documento}}
	\flushleft
	\normalsize
	Il presente documento ha lo scopo di presentare una visione globale della Specifica dei Requisiti del progetto Montani WebSite. La struttura del documento è quella suggerita dallo standard ANSI/IEEE  830 noto come SRS (Software Requirements Specifications).
	\normalsize
	 \subsection{\textbf{Obiettivo}} 
	\flushleft
	\normalsize
	Lo scopo del presente documento è di rappresentare, nel modo più preciso, completo, coerente, non  ambiguo e comprensibile, i requisiti relativi alla ridefinizione delle modalità di fruizione dei servizi scolastici mediante sito web dell'Istituto Tecnico Tecnologico ITT Montani di Fermo.\\ 
	Per specifica dei requisiti si intende l’elicitazione di tutti i requisiti utente e di sistema del sistema informativo senza specificare le metodologie applicate per la risoluzione di essi.
	
	\subsection{\textbf{Campo di applicazione}}
	L'obiettivo del presente progetto è di realizzare un sito web per l'Istituto Tecnico Tecnologico Montani di Fermo. La scuola dispone già di un sito web, il quale non rispetta i requisiti grafici e funzionali definiti dal modello standard di siti web scolastici realizzato dal Team per la Trasformazione Digitale su richiesta del Ministero dell'Istruzione. Il progetto si basa sulla metodologia, gli strumenti e il design system di Designers Italia e a sua volta, contribuisce ad alimentare il design system della Pubblica Amministrazione mettendo a disposizione di tutte le amministrazioni componenti e pattern elaborati. Per ulteriori informazioni consultare \url{https://docs.italia.it/italia/designers-italia/design-scuole-docs/it/master/progetto-siti-web-delle-scuole.html} e \url{https://www.funzionepubblica.gov.it/articolo/dipartimento/26-11-2009/direttiva-n-8-2009}. Partendo dallo studio della realtà pre-esistente è stato necessario ridefinire le modalità di navigazione e di interfacciamento con l'utente nel sito web. 
	Uno dei campi di applicazione del sito web è l'accessibilità, ovvero la capacità del sistema informativo di erogare informazioni rispettando le linee guida sull’accessibilità degli strumenti informatici secondo quanto descritto nell’articolo 11 della legge n. 4/2004.
	
	
	\subsection{\textbf{Definizioni, acronimi e abbreviazioni}}
	\begin{tabular}{ |p{3cm}|p{8cm}|  }
	\hline
	\textbf{Termine}& \textbf{Definizione}\\
	\hline
	Studente   & Studente che frequenta l'Istituto  \\
	\hline
	Genitori & Genitori degli Studenti \\
	\hline
	Personale tecnico amministrativo & Presidenza, Amministratori, Figure Amministrative \\
	\hline
	Insegnanti   &  Professionista nel campo dell'istruzione, operante nell'Istituto.  \\
	\hline
	Visitatore  & Soggetto non correlato all'Istituto in alcun modo se non per attività di orientamento. \\	
	\hline
	Header & Intestazione del sito presente in ogni pagina. \\
	\hline
	Forum & Luogo di scambio di messaggi visibile agli utenti registrati.\\
		\hline
	Topic & Argomento di discussione.\\
		\hline
	Post & Messaggio della discussione. \\
		\hline
	Repository & Archivio informazioni\\
	\hline
\end{tabular}
	
	\subsection{\textbf{Sinonimi}}
\begin{tabular}{ |p{3cm}|p{8cm}|  }
	\hline
	\textbf{Termine}& \textbf{Sinonimi}\\
	\hline
	Studente   & Alunno, Allievo  \\
	\hline
	Genitore   &  Parenti, Affidatari \\
	\hline
	Insegnanti  & Docenti\\	
	\hline
	Visitatore   & Studente non iscritto.  \\
	\hline
	Segreteria   & Personale della segreteria.  \\
\hline
\end{tabular}
	
	\subsection{\textbf{Struttura del documento e riferimenti}}
	Nella sezione 2 di questo documento verranno illustrate le interfacce di sistema e gli utenti coinvolti.\\
	Nella sezione 3 di questo documento verranno illustrate le specifiche funzionali e non funzionali del sistema.\\
	Questo documento si riferisce al project charter con ID Progetto Montani.\\
	Per la programmazione delle attività di lavoro fare riferimento ad Allegato B e per l'analisi dei rischi all'allegato C.\\
	
	\clearpage
	
	\section{\textbf{Descrizione generale}}
     Questa sezione descrive le funzionalità generale del prodotto, le interfacce e gli utenti del sistema.

	\subsection{\textbf{Inquadramento}}
	Questo progetto mira a ridefinire il modello di fruizione dei servizi scolastici dell'Istituto T.T. Montani di Fermo attraverso il suo sito web.
	
	Il sito web è accessibile da qualsiasi device connesso ad internet, attraverso l'utilizzo di un browser web e digitando l'indirizzo istitutomontani.edu.it.\\
	Al caricamento della pagina verrà mostrata l'homepage, dalla quale si potrà scegliere il facilmente il servizio che l'utente sta cercando. 
	
	\subsubsection{\textbf{Interfaccia utente}}
	L’interfaccia sistema/utente è stata realizzata attraverso un sito web il quale utilizza il CSM Wordpress. WordPress è una piattaforma software di content management system (CMS) che, operando lato server in un database, consente la creazione di un sito Internet formato da contenuti testuali o multimediali, gestibili ed aggiornabili in maniera dinamica; facendo uso di codice HTML CSS e JavaScript. Wordpress mette a disposizione anche plugin e temi per la personalizzazione del sito web. 
	
	\subsubsection{\textbf{Interfaccia software}}
	Come spiegato nella sezione 2.1.1 il sito web è stato realizzato attraverso il software di gestione di contenuti (CMS) Wordpress. Wordpress necessita di un database per la memorizzazione di tutte le informazioni relative al sito e un Web Server Apache.  
	
	\subsubsection{\textbf{Interfaccia hardware}}
	Per la mantenimento del web server e del database viene utilizzato il servizio hosting Aruba. Il dominio (istitutomontani.edu.it) segue le normative relative agli indirizzi istituzionali.
	
	\subsubsection{\textbf{Interfaccia di comunicazione}}
	Il sito si interfaccia con la piattaforma Spaggiari per le comunicazioni scolastiche.	

	\subsection{\textbf{Definizione degli utenti}}
\begin{tabular}{ |p{3cm}|p{8cm}|  }
	\hline
	\textbf{Tipo}& \textbf{Descrizione}\\
	\hline
	Studente   & Gli studenti possono visionare le informazioni relative ad orari. \\
	\hline
	Genitore   & I genitori possono visionare le informazioni relative ad orari.
	\\
	\hline
	Amministratori  & Gli amministratori hanno completo accesso alle funzionalità del CSM.\\	
	\hline
	Docente & I docenti hanno accesso al lato gestionale del CSM e wordpress e si occupano di inserire e modifica progettualità. Non possono modificare notizie.\\
	\hline
	Visitatore & I soggetti non iscritti al sito.\\
	\hline
	Segreteria & Personale della segreteria.\\
\hline
\end{tabular}

	\subsection{\textbf{Requisiti da analizzare in futuro}}
	Pop-up nella home page.	
\clearpage
	\Large \section{\textbf{Specifiche funzionali e non funzionali del sistema}} 

\subsection{\textbf{Requisiti funzionali}}
\normalsize
\flushleft
I requisiti funzionali sono stati suddivisi per categoria di utente (Fare riferimento alla sezione 2.2 per maggiori informazioni):
\begin{itemize}
	\item \textbf{Visitatore}:
	\begin{itemize}
		\item Connessione al sito e Visualizzazione della home page.
		\item Visualizzazione delle pagine delle articolazioni scolastiche.
		\item Visualizzazione delle comunicazioni e notizie scolastiche.
		\item Visualizzazione delle norme dell'Istituto.
		\item Visualizzazione della Storia dell'Istituto.
		\item Visualizzazione delle progettualità svolte.	
	\end{itemize}
	\item \textbf{Studente}: Gli studenti possiedono gli stessi requisiti funzionali dei Visitatori
	\begin{itemize}
		\item Visualizzazione degli orari scolastici.
		\item Visualizzazione delle assenza docenti.	
	\end{itemize}		
	\item \textbf{Genitori}: I requisiti funzionali coincidono con quelli del Visitatore.
	\item \textbf{Segreteria}:
		\begin{itemize}
			\item Caricamento dei libri di testo.	
	\end{itemize}
	\item \textbf{Docente}: I docenti possiedono gli stessi requisiti funzionali dei Visitatori e Studenti
	\begin{itemize}
		\item Caricamento di programmazioni scolastiche e progettualità adoperati.	
	\end{itemize}
	\item \textbf{Amministratore}: I docenti possiedono gli stessi requisiti funzionali dei Visitatori, Studenti, Docenti.
	\begin{itemize}
		\item Aggiunta di pagine, articoli, gestione back-end e accesso completo al database.
		\item Caricamento della modulistica.	
	\end{itemize}
	\item Il sito web deve essere disponibile a tutti i dispositivi dotati di connessione internet e deve prevedere una home di accesso dalle quale venga effettuata una panoramica dei servizi.
	\item Il sito web deve presentare un rifacimento sotto l'aspetto grafico al fin di modernizzare l'interfaccia attuale.
	\item Il sito web deve prevedere funzionalità aggiuntive come l'inserimento delle progettualità o l'aggiornamento degli orari.
	\item Il nuovo sito deve presentare la stessa offerta formativa presentata nella versione precedente del sito web.
	\item il sito deve presentare un interfacciamento attraverso la piattaforma Spaggiari per la visualizzazione delle circolari e dell'albo online.
	\item Il sito deve presentare una panoramica della storia dell'istituto aggiornata.
	\item Il sito web deve prevedere un sistema di login che identifichi attraverso un account la categoria di utente connessa.
	\end{itemize}
	\textbf{Requisiti funzionali aggiunti dopo prima revisione}
\begin{itemize}
	\item Il sito web deve prevedere una tabella orari standardizzata e facilmente accessibile..
	\item L'header del sito web deve presentare un logo più grande.
	\item Il sito web deve presentare dei menù in base al tipo di utente.
	\item Il sito web deve presentare una sezione notizie organizzata.
	\item Il sito web deve prevedere un Forum come luogo di incontro e discussione tra Studenti e Docenti.
	\item Il sito web deve presentare una repository di progetti (PCTO).
	\item Il sito web deve presentare Mappa dei plessi.
\end{itemize}
	\textbf{Requisiti funzionali aggiunti dopo seconda revisione}
\begin{itemize}
	\item Il sito web deve prevedere un sistema di login semplice (APi Auth di Google).
	\item Il sito web deve prevedere uno slide-show per le articolazioni che metta in luce tuttei i percorsi di studio scolastici. 
\end{itemize}
\subsubsection{\textbf{Caso d'uso: Connessione al sito e Visualizzazione della home page}}
\begin{tabular}{ |p{3cm}|p{9cm}|  }
	\hline
	\multicolumn{2}{|c|}{\textbf{RF - 001}} \\
	\hline
	Specifica& Connessione alla home page del sito\\
	\hline
	Attori& Visitatore/Studente/Docente/Genitore/Amministratore\\
	\hline
	Pre-Condizioni& L'attore deve predisporre di un dispositivo connesso alla rete internet e di un software browser\\
	\hline
	Scenario principale& \begin{enumerate}
		\item L'attore apre il broswer
		\item L'attore inserisce nella barra di ricerca l'indirizzo "istitutomontani.edu.it"
		\item La pagina home viene caricata con successo
	\end{enumerate}\\
	\hline
	Post-Condizioni& L'attore può visitare la home page del sito Montani\\
	\hline
\end{tabular}
\clearpage
\subsubsection{\textbf{Caso d'uso: Connessione alle sotto pagine del sito}}
\begin{tabular}{ |p{3cm}|p{9cm}|  }
	\hline
	\multicolumn{2}{|c|}{\textbf{RF - 002}} \\
	\hline
	Specifica& Connessione alle pagine secondario\\
	\hline
	Attori& Visitatore/Studente/Docente/Genitore/Amministratore\\
	\hline
	Pre-Condizioni& L'attore deve essere connesso alla home o al sito.\\
	\hline
	Scenario principale& \begin{enumerate}
		\item L'attore procede a visualizzare le pagine presenti nelle due barre in alto nel sito (header)
		\item L'attore clicca la sotto pagina da visitare. (Nella prima barra sono presenti le comunicazioni, informazioni e la seconda barra è dedicata agli indirizzi di studio)
		\item La pagina viene caricata con successo.
	\end{enumerate}\\
	\hline
	Post-Condizioni& L'attore si trova nella pagina scelta.\\
	\hline
\end{tabular}
\subsubsection{\textbf{Caso d'uso: Inserimento di nuovi articoli o post}}
\begin{tabular}{ |p{3cm}|p{9cm}|  }
	\hline
	\multicolumn{2}{|c|}{\textbf{RF - 003}} \\
	\hline
	Specifica& Inserimento di nuovi articoli o post\\
	\hline
	Attori& Amministratore\\
	\hline
	Pre-Condizioni& L'attore deve aver effettuato il login nella sezione admin Wordpress.\\
	\hline
	Scenario principale& \begin{enumerate}
		\item L'attore sceglie "New" dalla barra in alto presente in ogni pagina del sito
		\item L'attore sceglie la tipologia di post da effettuare ed inserisce il testo/media dall'apposito editor.
		\item L'attore clicca il tasto "Publish" a sinistra nell'editor.
	\end{enumerate}\\
	\hline
	Post-Condizioni& Il post/articolo viene pubblicato.\\
	\hline
\end{tabular}
\subsubsection{\textbf{Caso d'uso: Login account}}
\begin{tabular}{ |p{3cm}|p{9cm}|  }
	\hline
	\multicolumn{2}{|c|}{\textbf{RF - 004}} \\
	\hline
	Specifica& Accesso al proprio account\\
	\hline
	Attori& Studente/Docente/Amministratore\\
	\hline
	Pre-Condizioni& L'attore deve possedere un account.\\
	\hline
	Scenario principale& \begin{enumerate}
		\item L'attore clicca sul tasto Accedi in alto a destra nel sito.
		\item il sistema visualizza i campi dove inserire username e password dell'utente.
		\item L'attore inserisce username e password.
		\item Il sistema verifica se username e password sono corretti.
		\item Se username e password sono corretti il sistema visualizza la homepage.
	\end{enumerate}\\
	\hline
	Post-Condizioni& L'attore ha avuto accesso al sito.\\
	\hline
	Scenario secondario& L'attore ha sbagliato password o non le sue credenziali non esistono.\\
	\hline
	Post-Condizioni& L'attore chiede assistenza alla scuola.\\
	\hline
\end{tabular}
\subsubsection{\textbf{Caso d'uso: Inserimento Progettualità}}
\begin{tabular}{ |p{3cm}|p{9cm}|  }
	\hline
	\multicolumn{2}{|c|}{\textbf{RF - 005}} \\
	\hline
	Specifica& Inserimento Progettualità\\
	\hline
	Attori& Studente/Docente/Amministratore\\
	\hline
	Pre-Condizioni& L'attore deve possedere un account.\\
	\hline
	Scenario principale& \begin{enumerate}
		\item L'attore clicca sul tasto Accedi in alto a destra nel sito.
		\item il sistema visualizza i campi dove inserire username e password dell'utente.
		\item L'attore inserisce username e password.
		\item Il sistema verifica se username e password sono corretti.
		\item Se username e password sono corretti il sistema visualizza la homepage.
	\end{enumerate}\\
	\hline
	Post-Condizioni& L'attore ha avuto accesso al sito.\\
	\hline
	Scenario secondario& L'attore ha sbagliato password o non le sue credenziali non esistono.\\
	\hline
	Post-Condizioni& L'attore chiede assistenza alla scuola.\\
	\hline
\end{tabular}
\subsubsection{\textbf{Caso d'uso: Visualizzazione progettualità}}
\begin{tabular}{ |p{3cm}|p{9cm}|  }
	\hline
	\multicolumn{2}{|c|}{\textbf{RF - 006}} \\
	\hline
	Specifica& Visualizzazione progettualità\\
	\hline
	Attori& Studente/Docente/Amministratore\\
	\hline
	Pre-Condizioni& L'attore deve possedere un account (fornito dalla scuola).\\
	\hline
	Scenario principale& \begin{enumerate}
		\item L'attore clicca sul tasto Accedi in alto a destra nel sito.
		\item il sistema visualizza i campi dove inserire username e password dell'utente.
		\item L'attore inserisce username e password.
		\item Il sistema verifica se username e password sono corretti.
		\item Se username e password sono corretti il sistema visualizza la homepage.
	\end{enumerate}\\
	\hline
	Post-Condizioni& L'attore ha avuto accesso al sito.\\
	\hline
	Scenario secondario& L'attore ha sbagliato password o non le sue credenziali non esistono.\\
	\hline
	Post-Condizioni& L'attore chiede assistenza alla scuola.\\
	\hline
\end{tabular}
\subsubsection{\textbf{Caso d'uso: Visualizzazione orari scolastici}}
\begin{tabular}{ |p{3cm}|p{9cm}|  }
	\hline
	\multicolumn{2}{|c|}{\textbf{RF - 007}} \\
	\hline
	Specifica& Visualizzazione orari scolastici\\
	\hline
	Attori& Studente/Docente/Amministratore\\
	\hline
	Pre-Condizioni& L'attore deve possedere un account (fornito dalla scuola).\\
	\hline
	Scenario principale& \begin{enumerate}
		\item L'attore clicca sul tasto Accedi in alto a destra nel sito.
		\item il sistema visualizza i campi dove inserire username e password dell'utente.
		\item L'attore inserisce username e password.
		\item Il sistema verifica se username e password sono corretti.
		\item Se username e password sono corretti il sistema visualizza la homepage.
	\end{enumerate}\\
	\hline
	Post-Condizioni& L'attore ha avuto accesso al sito.\\
	\hline
	Scenario secondario& L'attore ha sbagliato password o non le sue credenziali non esistono.\\
	\hline
	Post-Condizioni& L'attore chiede assistenza alla scuola.\\
	\hline
\end{tabular}
\subsubsection{\textbf{Caso d'uso: Creazione di un topic nel Forum}}
\begin{tabular}{ |p{3cm}|p{9cm}|  }
	\hline
	\multicolumn{2}{|c|}{\textbf{RF - 008}} \\
	\hline
	Specifica& Creazione di un topic nel Forum\\
	\hline
	Attori& Studente/Docente/Amministratore\\
	\hline
	Pre-Condizioni& L'attore deve possedere un account (fornito dalla scuola).\\
	\hline
	Scenario principale& \begin{enumerate}
		\item L'attore clicca sul tasto Forum sulla barra dell'header.
		\item Il sistema visualizza l'interfaccia Forum.
		\item L'attore sceglie un Forum tra quelli elencati.
		\item Il sistema visualizza i Topic esistenti in quella categoria.
		\item L'attore clicca su Nuovo Topic ed inserisce l'argomento della discussione.
	\end{enumerate}\\
	\hline
	Post-Condizioni& Il topic viene aggiunto con successo.\\
	\hline
\end{tabular}
\subsubsection{\textbf{Caso d'uso: Creazione di un post nel Forum}}
\begin{tabular}{ |p{3cm}|p{9cm}|  }
	\hline
	\multicolumn{2}{|c|}{\textbf{RF - 009}} \\
	\hline
	Specifica& Creazione di un post nel Forum\\
	\hline
	Attori& Studente/Docente/Amministratore\\
	\hline
	Pre-Condizioni& L'attore deve possedere un account (fornito dalla scuola).\\
	\hline
	Scenario principale& \begin{enumerate}
		\item L'attore clicca sul tasto Forum sulla barra dell'header.
		\item Il sistema visualizza l'interfaccia Forum.
		\item L'attore sceglie un Forum tra quelli elencati.
		\item Il sistema visualizza i Topic esistenti in quella categoria.
		\item L'attore clicca su un Topic.
		\item Il sistema visualizza i post pubblicati nel Topic.
		\item L'attore inserisce una nuova risposta nel form in basso.
	\end{enumerate}\\
	\hline
	Post-Condizioni& Il post viene aggiunto con successo.\\
	\hline
\end{tabular}
\subsubsection{\textbf{Caso d'uso: Caricamento di programmazioni}}
\begin{tabular}{ |p{3cm}|p{9cm}|  }
	\hline
	\multicolumn{2}{|c|}{\textbf{RF - 010}} \\
	\hline
	Specifica& Caricamento di programmazioni\\
	\hline
	Attori& Docente\\
	\hline
	Pre-Condizioni& L'attore deve possedere un account (fornito dalla scuola).\\
	\hline
	Scenario principale& \begin{enumerate}
		\item da definire.
	\end{enumerate}\\
	\hline
	Post-Condizioni& Il post viene aggiunto con successo.\\
	\hline
\end{tabular}
\subsubsection{\textbf{Caso d'uso: Caricamento di modulistica e libri di testo}}
\begin{tabular}{ |p{3cm}|p{9cm}|  }
	\hline
	\multicolumn{2}{|c|}{\textbf{RF - 011}} \\
	\hline
	Specifica& Creazione di un post nel Forum\\
	\hline
	Attori& Segreteria/Amministratore\\
	\hline
	Pre-Condizioni& L'attore deve possedere un account (fornito dalla scuola).\\
	\hline
	Scenario principale& \begin{enumerate}
		\item da definire.
	\end{enumerate}\\
	\hline
	Post-Condizioni& Il post viene aggiunto con successo.\\
	\hline
\end{tabular}
\normalsize
\flushleft
\vspace{4mm} 
\subsection{\textbf{Requisiti non funzionali}}
In questa sezione si sintetizzano tutte le caratteristiche non funzionali emerse dalla fase di  analisi dei vincoli di diversa natura raccolti durante le interviste e lo studio dell'ambito d'applicazione. Queste caratteristiche definiscono tutte le funzionalità “esterne” al sistema:
\vspace{4mm} \vspace{4mm} 
\begin{tabular}{ |p{3cm}|p{8cm}|  }
	\hline
	\multicolumn{2}{|c|}{\textbf{RNF - 001}} \\
	\hline
	Specifica&Portabilità / Responsive\\
	\hline
	Metrica & Dispositivi\\
	\hline
	Descrizione&Il sito web deve essere accessibile e visualizzabile da ogni dispositivo con connessione internet ed un browser web. Il sito deve essere portabile anche su altri Web Server esportando il database MySQL. Il CSM Wordpress supporta tutti i browser moderni. Il sito web deve essere disponibile su dispositivi di natura diversa.\\
	\hline
\end{tabular}\\
\vspace{4mm} 
\begin{tabular}{ |p{3cm}|p{8cm}|  }
	\hline
	\multicolumn{2}{|c|}{\textbf{RNF - 002}} \\
	\hline
	Specifica&Usabilità \\
	\hline
	Metrica & Tempo di lettura articoli, Facilità di apertura\\
	\hline
	Descrizione&Il sito web deve essere comprensibile e facilmente leggibile. Il tempo medio di lettura di un articolo o post deve essere inferiore ai 2 minuti.\\ Il sito web deve prevedere dei percorsi indicizzati per categoria di utente.\\
	\hline
\end{tabular}\\
\vspace{4mm} 
\begin{tabular}{ |p{3cm}|p{8cm}|  }
	\hline
	\multicolumn{2}{|c|}{\textbf{RNF - 003}} \\
	\hline
	Specifica &Implementazione\\
	\hline
	Metrica &  Versione sistema\\
	\hline
	Descrizione&Il sito web deve essere implementato attraverso il CSM Wordpress. Il sistema deve soddisfare i requisiti richiesti da Wordpress (PHP versione 7.4 o superiore.
	MySQL versione 5.7 o superiore O MariaDB versione 10.2 o superiore.
	HTTPS compatibilità protocollo.) Inoltre come definito dal Ministero dell'Istruzione tutti i siti di nuova realizzazione, devono utilizzare almeno la versione 4.01 dell'HTML o la versione 1.0 dell'XHTML, entrambe con DTD (Definizione del Tipo di Documento) di tipo Strict.\\
	\hline
\end{tabular}\\
\vspace{4mm} 
\begin{tabular}{ |p{3cm}|p{8cm}|  }
	\hline
	\multicolumn{2}{|c|}{\textbf{RNF - 004}} \\
	\hline
	Specifica&Legislativo\\
	\hline
	Metrica &  -\\
	\hline
	Descrizione&Le norme legislative da rispettare saranno esplicitate direttamente attraverso il sito.\\
	\hline
\end{tabular}\\
\vspace{4mm} 
\begin{tabular}{ |p{3cm}|p{8cm}|  }
	\hline
	\multicolumn{2}{|c|}{\textbf{RNF - 005}} \\
	\hline
	Specifica&Temporale \\
	\hline
	Metrica &  -\\
	\hline
	Descrizione&La prima versione del sito web funzionante deve essere consegnata entro il 28/02/2022.\\
	\hline
\end{tabular}\\
\vspace{4mm} 
\begin{tabular}{ |p{3cm}|p{8cm}|  }
	\hline
	\multicolumn{2}{|c|}{\textbf{RNF - 006}} \\
	\hline
	Specifica&Budget \\
	\hline
	Metrica &  -\\
	\hline
	Descrizione&Allo stato attuale non è presente un budget per lo sviluppo del sito.\\
	\hline
\end{tabular}\\
\vspace{4mm} 
\begin{tabular}{ |p{3cm}|p{8cm}|  }
	\hline
	\multicolumn{2}{|c|}{\textbf{RNF - 007}} \\
	\hline
	Specifica&Scalabilità \\
	\hline
	Metrica &  -\\
	\hline
	Descrizione&Il sistema prevede di essere utilizzato negli anni a seguire, sarà quindi prevista un'evoluzione del sito nel tempo.\\
	\hline
\end{tabular}\\
\vspace{4mm} 
\begin{tabular}{ |p{3cm}|p{8cm}|  }
	\hline
	\multicolumn{2}{|c|}{\textbf{RNF - 008}} \\
	\hline
	Specifica&Standard \\
	\hline
	Metrica &  -\\
	\hline
	Descrizione&Il sito rispetta gli standard definiti dal Team del Ministero della Trasformazione Digitale su commissione del Ministero dell'Istruzione in ambito di siti web ministeriali.
	Il sito web deve supportare Google Analytics per il monitoraggio del traffico. Il sito web deve supportare SEO per l'indicizzazione nel web.\\
	\hline
\end{tabular}\\
\vspace{4mm} 
\begin{tabular}{ |p{3cm}|p{8cm}|  }
	\hline
	\multicolumn{2}{|c|}{\textbf{RNF - 009}} \\
	\hline
	Specifica&Etico \\
	\hline
	Metrica &  -\\
	\hline
	Descrizione&Il sito web deve essere leggibile da web reader per soggetti ipovedenti.\\ Il sito web deve prevedere un sistema di traduzione dell'intero sito.\\
	\hline
\end{tabular}\\

\section{\textbf{Ulteriori specifiche}}
Ulteriori specifiche, se necessarie, verranno aggiunte nelle prossime versioni del seguente documento.
\end{document}
	
    