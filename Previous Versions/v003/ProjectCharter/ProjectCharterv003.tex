\documentclass{article}



\usepackage{graphicx}                       % Pacchetti

\usepackage[italian]{babel}

\graphicspath{ {./images/} }

\usepackage{imakeidx}

\usepackage{hyperref}%



\makeindex[columns=1, title=Tavola dei contenuti, intoc]



\begin{document}

	

	

	\begin{titlepage}

		\begin{center}

			\huge\textbf{Montani Website}\\

			\Large\textbf{5 INB}\\

			\Large \textbf{Project Charter}\\

			\vspace{4cm}

			\large Project Manager: \textbf{Boussoufa Yacine}\\

			\large Data: \textbf{12/02/2022}\\

			\large Versione: \textbf{0.3}\\

		\end{center}

	\end{titlepage}

	

	\clearpage

	

	\begin{tabular}{ |p{1cm}|p{4cm}|p{3cm}|p{2cm}|  }

		\hline

		\multicolumn{4}{|c|}{Cronologia delle revisioni} \\

		\hline

		ID& Cambiamenti &Data di creazione&Autore\\

		\hline

		1   & Creazione    &07/02/2022&   Capretti Mattia\\

		\hline

		2   & Correzioni e Sviluppo    &08/02/2022&   Camilletti Samuele\\

		\hline

		3   & Correzioni e Sviluppo    &11/02/2022&   Capretti Mattia\\

		\hline

	\end{tabular}

	

	\clearpage


	\tableofcontents
	

	\index{generate}

	\clearpage
	

	%Stampa del titolo, autore e data

	

	

	\textbf{{\fontsize{5mm}{10mm}\selectfont \section{Background} }} 

	\begin{flushleft}

		L'Istituto Tecnico Tecnologico Statale “G. e M. MONTANI” di Fermo ha commissionato una modernizzazione grafica e funzionale del sito web dell'Istituto, il quale non rispetta i requisiti grafici e funzionali definiti dal modello standard di siti web scolastici realizzato dal Team per la Trasformazione Digitale su richiesta del Ministero dell’Istruzione.

		

	\end{flushleft}

	\vspace{3mm}

	\textbf{{\fontsize{5mm}{10mm}\selectfont \section{Obiettivo del progetto} }} 

	\begin{flushleft}

		L'obiettivo del presente progetto è di realizzare un sito web per l'Istituto Tecnico Tecnologico Montani di Fermo, al fine di risolvere i problemi di accessibilità e trasparenza riscontrati dalle famiglie degli studenti, dai professori e dal personale dell'Istituto. 

		Le problematiche riscontrate sono maggiormente dovute ad una interfaccia grafica poco intuitiva e non più in regola con le direttive del Ministero.   

		Il restyling deve prevedere:

		\begin{itemize}

			\item \textbf{Home Page}: miglioramento della home-page intento a rendere più intuitivo l' approccio al sito. 

			\item \textbf{Pagine degli indirizzi}: miglioramento delle pagine contenenti le informazioni riguardanti gli indirizzi e le articolazioni scolastiche.

			\item \textbf{Pagine non funzionanti}: risoluzione dei problemi legati ai link non funzionanti o che indirizzano a pagine non esistenti.

			\item \textbf{Login e informazioni targhettizzate}: Consentire solo agli utenti loggati di visualizzare informazioni legate alle ore dei professori.

			\item \textbf{Aggiunta di funzionalità aggiuntive}: Il restyling prevede anche l'aggiunta di numerose funzionalità descritte nell'allegato A.

			

		\end{itemize}

		\vspace{2mm}

		

		\subsection{Progettazione preliminare}

		Il sito Web verrà realizzato attraverso l'utilizzo del CSM Wordpress. Precedentemente alla realizzazione di questo progetto, l'Istituto Tecnico Tecnologico statale “G. e M. MONTANI” è già in possesso un sito web e di conseguenza non sarà sviluppato ex-novo.

		Per la specifica dei requisiti e relativi casi d'uso consultare \textbf{allegato "A"}.\\ 

		\vspace{2mm} 

	\end{flushleft}

	\vspace{1cm}

	\textbf{{\fontsize{5mm}{10mm}\selectfont \section{Ambito del progetto (SCOPE)} }} 

	\begin{flushleft}

		Definizione dei deliverable:

		\begin{itemize}

			\item Project management: Analisi del sito web in relazioni ai requisiti utente e di sistema.

			\item Documentazione: redazione documento SRS e project charter.

			\item Realizzazione: prototipo grafico del sito.

			\item Test: confronto con il cliente.

		\end{itemize}

		

		Definizione delle milestone:

		\begin{itemize}
            
            \item Firma del Project Charter
            
			\item Accordo con il cliente per l'interfaccia grafica.

			\item Realizzazione interfaccia grafica definitiva.

			\item Caricamento dei materiali relativo alle pagine e modellazione database.

			\item Consegna della versione finale del sito.

		\end{itemize}
        
        Le fasi di deliverable e milestone verranno ripetute ad ogni ciclo di sviluppo evolutivo.\\
        
    
		\subsection{In ambito}

		Nel progetto è incluso:\\

	    \begin{itemize}

			\item \textbf{Restyling Grafico}: miglioramento dell'interfaccia grafica intento a rendere più intuitivo l'approccio al sito. 

			\item \textbf{Rimodulazione dei servizi offerti}: miglioramento della struttura del sito web.

			\item \textbf{Login System}: Consentire solo agli utenti loggati di visualizzare informazioni legate alle ore dei professori.

			\item \textbf{Back-end}: Sviluppo del database contenente progettualità, modulistica, orari e libri di testo.
			
			\item \textbf{Forum}: Sviluppo di un forum dedicato al confronto tra  studenti e docenti.

		\end{itemize}

		\subsection{Fuori ambito}

		Nel progetto non è incluso:

		\begin{itemize}

			\item Realizzazione ex-novo dell'offerta formativa: i contenuti del sito sono considerati validi.

		\end{itemize}

		\vspace{2mm}

		\subsection{Programmazione della tempistica preliminare}

		\begin{tabular}{ |p{6cm}|p{4cm}|  }

			\hline

			Descrizione del prodotto da consegnare &Data obiettivo\\

			\hline

			Redazione documentazione project charter &09/02/2022\\

			\hline

			Realizzazione prototipo interfaccia grafica &09/02/2022\\

			\hline

			Confronto con il committente &09/02/2022\\

			\hline

			Confronto con il committente &14/02/2022\\

			\hline

			Debug e consegna &28/02/2022\\

			\hline

		\end{tabular}

		\vspace{3mm}

		\textbf{{\fontsize{5mm}{10mm}\selectfont \section{Criteri di successo} }} 

		Il sito web soddisfare i criteri stabiliti nell'allegato relativo alla specifica dei requisiti:

		\begin{itemize}

			\item Funzionali: esegue i requisiti funzionali prioritari del progetto.

			\item Non funzionali: rispetta i requisiti non funzionali del progetto.

			\item Esterni: è conforme con le norme sulla privacy attualmente in vigore.

			\item Organizzativi: è conforme alle caratteristiche organizzative dell'Istituto.

		\end{itemize}

		\vspace{3mm}

		\textbf{{\fontsize{5mm}{10mm}\selectfont \section{Assunzioni e vincoli iniziali} }} 

		\begin{tabular}{ |p{1cm}|p{2cm}|p{6cm}|  }

			\hline

			ID&Tipo &Descrizione\\

			\hline

			0&Temporale&Scadenza 28/02/2022\\

			\hline

			1&Economico&Inesistenza di budget\\

			\hline

			2&Ambientale&Ambiente di sviluppo - CSM Wordpress-Joomla\\

			\hline

			3&Legislativo&Politiche relative all'uso delle informazioni degli utenti\\

			\hline

		\end{tabular}

		\vspace{3mm}

		\textbf{{\fontsize{5mm}{10mm}\selectfont \section{Team e comunicazione} }}

		\subsection{Definizione degli stakeholders}

		\begin{itemize}

			\item Promotore: Istituto Tecnico Tecnologico Statale “G. e M. MONTANI” di Fermo

			\item Cliente: Istituto Tecnico Tecnologico Statale “G. e M. MONTANI” di Fermo

			\item Beneficiario: Istituto Tecnico Tecnologico Statale “G. e M. MONTANI” di Fermo

			\item Team di progetto:

			\begin{itemize}

				\item Project Manager: Boussoufa Yacine

				\item Analyst: Capretti Mattia

				\item Programmer: Enrique Nucci

				\item Debugger: Camilletti Samuele

			\end{itemize}

			\item Soggetti esterni all'organizzazione legati alle risorse: Daniele Trasatti

			\item Soggetti condizionanti: famiglie degli studenti, studenti, professori dell'istituto, personale ATA

		\end{itemize}

		\subsection{Parti interessate, Ruoli e responsabilità}

		\begin{tabular}{ |p{4cm}|p{4cm}|p{4cm}|  }

			\hline

			Parte interessata&Ruolo &Responsabilità\\

			\hline

			Boussoufa Yacinea&Project Manager&Alta\\

			\hline

			Capretti Mattia&Analyst-debugger&Alta\\

			\hline

			Enrique Nucci&Programmer&Alta\\

			\hline

			Camilletti Samuele&Analyst&Alta\\

			\hline

			Istituto Tecnico Tecnologico Statale “G. e M. MONTANI” di Fermo&Committente&Media\\

        	\hline

        	Studente   & Utente del sito & Bassa \\

        	\hline

        	Genitore   & Utente del sito & Bassa \\

        	\hline

        	Amministratori  & Amministratore del sito & Bassa\\	

        	\hline

        	Docente & Utente del sito & Bassa\\

        	\hline

		\end{tabular}

		\vspace{3mm}

		\subsection{Piano di comunicazione iniziale}

		\begin{tabular}{ |p{3cm}|p{3.5cm}|p{2cm}|p{1.5cm}|p{2cm}|  }

			\hline

			Comunicazione&Descrizione &Frequenza&Formato&Destinatari\\

			\hline

			Aggiornamento stato con il cliente&Confronto dei progressi con i requisiti utente&Ogni 2 giorni&Smart-working&Daniele Trasatti\\

			\hline

		\end{tabular}

		

		\textbf{{\fontsize{5mm}{10mm}\selectfont \section{Analisi dei rischi} }} 

		Per l'analisi e valutazione dei rischi consultare \textbf{allegato "B"}.

		\begin{tabular}{ |p{1cm}|p{2cm}|p{6cm}|p{2cm}|  }

			\hline

			ID&Tipo &Descrizione&Livello\\

			\hline

			0&Temporale&Tempo necessario per la realizzazione dell'interfaccia grafica insufficente&Basso\\	

			\hline

			1&Requisiti&Incertezza dei requisiti funzionali&Basso\\	

			\hline

			2&Team&Mancata comprensione nel team&Medio\\	

            \hline

			3&Team&Team senza le qualifiche necessarie&Basso\\	

            \hline

			2&Software&Impossibilità di utilizzare plugin a pagamento&Medio\\	

            \hline            

		\end{tabular}

		

		\vspace{12mm}

		\textbf{{\fontsize{5mm}{10mm}\selectfont \section{Autorità del progetto} }} 

		Sponsor: Istituto Tecnico Tecnologico Statale “G. e M. MONTANI” di Fermo\\

		Project Manager: Boussoufa Yacine\\

		Committente: Istituto Tecnico Tecnologico Statale “G. e M. MONTANI” di Fermo\\

		\vspace{2mm}

		\subsection{Allegati}

		\begin{itemize}

			\item Allegato A: SRS.

			\item Allegato B: Analisi e valutazione dei rischi.

			\item Allegato C: RBS.

			\item Allegato D: Gantt.

		\end{itemize}

		\subsection{Firme}

		Project Manager: Boussoufa Yacine\\

		Committente: Daniele Trasatti

		

	\end{flushleft}

	

	

	%_______________________________________________________________________________

	%fine del documento

\end{document}



